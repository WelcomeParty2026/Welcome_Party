\newpage

\section{入浴}  % 割り振りの名前を記入

\subsection{場所・人員配置}
\begin{table}[h]
  \centering
  \begin{minipage}[t]{0.48\columnwidth}
    \centering
    \caption{場所・該当グループ}
    \begin{tabular}{cl}
      \hline
      場所 & 宿泊棟 \\
      & 大浴場 \\
      & 小浴場 \\
      \hline
      グループ & 裏方 \\
      & 班付 \\
      \hline
    \end{tabular}
  \end{minipage}
  \begin{minipage}[t]{0.48\columnwidth}
    \centering
    \caption{役割分担}
    \begin{tabular}{ll}
      \hline
      \multicolumn{1}{c}{役割} & \multicolumn{1}{c}{担当者} \\
      \hline
      入浴整理・見張り(前半) & 落合,小谷,猫田,石田 \\
      入浴整理・見張り(後半) & 畠中,芳野,平本,加藤 \\
      龍馬管理(前半) & 浅島 \\
      龍馬管理(後半) & 花井 \\
      鍵管理 & 佐藤謙 \\
      \hline
    \end{tabular}
  \end{minipage}
\end{table}

\subsection{詳細タイムスケジュール}
\vspace{0.5em}
\fontsize{8pt}{12pt}\selectfont

\begin{longtable}[c]{|p{.04\textwidth}|p{.2\textwidth}|p{.25\textwidth}|p{.45\textwidth}|}
  \caption{スケジュール} \\
  \hline
  \multicolumn{1}{|c|}{時間} & \multicolumn{1}{c|}{担当} & \multicolumn{1}{c|}{内容} & \multicolumn{1}{c|}{備考} \\
  \hline
  \endfirsthead
  
  \hline
  \multicolumn{1}{|c|}{時間} & \multicolumn{1}{c|}{担当} & \multicolumn{1}{c|}{内容} & \multicolumn{1}{c|}{備考} \\
  \hline
  \endhead
  
  \hline
  \endfoot
  
  \hline
  \endlastfoot
  
  \begin{tpacol}
    19:00
  \end{tpacol}&
  \begin{tpbcol}
    班付
  \end{tpbcol}&
  \begin{tpccol}
    ベッドメイキング指導開始
  \end{tpccol}&
  \begin{tpdcol}
    各棟で指導 \par
    ベッドメイキング終わった人から入浴に誘導する
  \end{tpdcol}\\

  \begin{tpacol}
    
  \end{tpacol}&
  \begin{tpbcol}
    見張り(前半)
  \end{tpbcol}&
  \begin{tpccol}
    1回目入浴開始
  \end{tpccol}&
  \begin{tpdcol}
    大浴場(男性)・小浴場(女性) \par
    浴室を間違えないように,人数が多くなりすぎないように調整を行う
  \end{tpdcol}\\

  \begin{tpacol}
    19:55
  \end{tpacol}&
  \begin{tpbcol}
    見張り(前半)
  \end{tpbcol}&
  \begin{tpccol}
    忘れ物確認
  \end{tpccol}&
  \begin{tpdcol}
    忘れ物がないか軽く確認し,発見した場合は\#当日連絡に報告する
  \end{tpdcol}\\


  \begin{tpacol}
    20:00
  \end{tpacol}&
  \begin{tpbcol}
    見張り(後半)
  \end{tpbcol}&
  \begin{tpccol}
    2回目入浴開始
  \end{tpccol}&
  \begin{tpdcol}
    浴室を間違えないように,人数が多くなりすぎないように調整を行う
  \end{tpdcol}\\

  \begin{tpacol}
    20:40
  \end{tpacol}&
  \begin{tpbcol}
    見張り(後半)
  \end{tpbcol}&
  \begin{tpccol}
    入浴終了案内
  \end{tpccol}&
  \begin{tpdcol}
    
  \end{tpdcol}\\

  \begin{tpacol}
    21:00
  \end{tpacol}&
  \begin{tpbcol}
    全員
  \end{tpbcol}&
  \begin{tpccol}
    入浴終了
  \end{tpccol}&
  \begin{tpdcol}
    忘れ物がないかを確認する
  \end{tpdcol}\\

\end{longtable}
\normalsize

\newpage

\subsection{ベッドメイキング}
\begin{figure}[htbp]
  \centering
    \includegraphics[width = 10cm]{document/3_Deta/09_nyuyoku.png}
\end{figure}  


\subsection{備考}
\begin{itemize}
    \item 利用始めにドライヤーの有無と忘れ物がないか確認する
    \item 鍵の管理は厳重に行い,貸し借りの際はSlackで報告
    \item スタッフも原則21時までにお風呂に入っておく
    \item 準備が終わり次第,次の工程へ移行
    \item 入浴がスムーズに行えるように、誘導スタッフはタイミングを見計らって指示を出す
    \item 飲酒した場合は入浴できない
\end{itemize}
