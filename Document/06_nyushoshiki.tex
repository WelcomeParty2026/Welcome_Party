\newpage

\section{入所式} 

\subsection{場所・人員配置}
\begin{table}[h]
  \centering
  \begin{minipage}[t]{0.48\columnwidth}
    \centering
    \caption{場所・該当グループ}
    \begin{tabular}{cl}
      \hline
      場所 & 体育館 \\
      \hline
      グループ & 班付 \\
       & 先遣隊 \\
       & 後遣隊 \\
      \hline
    \end{tabular}
  \end{minipage}
  \begin{minipage}[t]{0.48\columnwidth}
    \centering
    \caption{役割分担}
    \begin{tabular}{ll}
      \hline
      \multicolumn{1}{c}{役割} & \multicolumn{1}{c}{担当者} \\
      \hline
      司会者 & 打越 \\
      教員の誘導 & 林 \\
      入所挨拶 & 芦田 \\
      鍵係 & 眞鍋 \\
      誘導 & 森田,仲村,高橋 \\
      \hline
    \end{tabular}
  \end{minipage}
\end{table}


\subsection{詳細タイムスケジュール}
\vspace{0.5em}
\fontsize{8pt}{12pt}\selectfont

\begin{longtable}[c]{|p{.04\textwidth}|p{.2\textwidth}|p{.25\textwidth}|p{.45\textwidth}|}
  \caption{スケジュール} \\
  \hline
  \multicolumn{1}{|c|}{時間} & \multicolumn{1}{c|}{担当} & \multicolumn{1}{c|}{内容} & \multicolumn{1}{c|}{備考} \\
  \hline
  \endfirsthead
  
  \hline
  \multicolumn{1}{|c|}{時間} & \multicolumn{1}{c|}{担当} & \multicolumn{1}{c|}{内容} & \multicolumn{1}{c|}{備考} \\
  \hline
  \endhead
  
  \hline
  \endfoot
  
  \hline
  \endlastfoot

  \begin{tpacol}
    14:40
  \end{tpacol}&
  \begin{tpbcol}
    班付
  \end{tpbcol}&
  \begin{tpccol}
    誘導
  \end{tpccol}&
  \begin{tpdcol}
    バスを降りたらそのままプレイルームに誘導する \par
    プレイルームの所定の場所で荷物を下ろし,しおりを持って体育館に誘導する \par
    トイレに行きたい人には体育館にトイレがあることを伝える
  \end{tpdcol}\\

  \begin{tpacol}
    14:50
  \end{tpacol}&
  \begin{tpbcol}
    班付
  \end{tpbcol}&
  \begin{tpccol}
    誘導完了
  \end{tpccol}&
  \begin{tpdcol}
    班ごとに並ばせ,人数を確認する \par
    班の新入生のメンバーが全員揃っていたら,司会者に報告する \par
    班のメンバーに,15:00に入所式が始まる旨を伝え,それまでは自由に話していていいと伝える
  \end{tpdcol}\\

  \begin{tpacol}
    
  \end{tpacol}&
  \begin{tpbcol}
    後遣隊
  \end{tpbcol}&
  \begin{tpccol}
    荷物運び
  \end{tpccol}&
  \begin{tpdcol}
    到着した旨を\#当日報告に連絡する \par
    荷物をプレイルームに運び,入所式に参加する
  \end{tpdcol}\\

  \begin{tpacol}
    
  \end{tpacol}&
  \begin{tpbcol}
    鍵係
  \end{tpbcol}&
  \begin{tpccol}
    プレイルームの鍵閉め
  \end{tpccol}&
  \begin{tpdcol}
    後遣隊が来次第,プレイルームの鍵を閉め,入所式に参加する
  \end{tpdcol}\\


  \begin{tpacol}
    15:00
  \end{tpacol}&
  \begin{tpbcol}
    
  \end{tpbcol}&
  \begin{tpccol}
    入所式開始
  \end{tpccol}&
  \begin{tpdcol}
    
  \end{tpdcol}\\

  \begin{tpacol}
    
  \end{tpacol}&
  \begin{tpbcol}
    司会
  \end{tpbcol}&
  \begin{tpccol}
    司会者の挨拶と代表の挨拶
  \end{tpccol}&
  \begin{tpdcol}
    代表は新入生に向けた挨拶をする
  \end{tpdcol}\\
  
  \begin{tpacol}
    15:05
  \end{tpacol}&
  \begin{tpbcol}
    司会
  \end{tpbcol}&
  \begin{tpccol}
    教員紹介(教員が自己紹介)
  \end{tpccol}&
  \begin{tpdcol}
    2~3分/人 \par
    マイクはリレー形式で回していただき,最後の教職員からマイクを受け取る \par
    司会は新歓に参加できない先生と遅れてくる先生の紹介をする
  \end{tpdcol}\\
  
  \begin{tpacol}
    15:35
  \end{tpacol}&
  \begin{tpbcol}
    施設職員
  \end{tpbcol}&
  \begin{tpccol}
    施設利用の説明
  \end{tpccol}&
  \begin{tpdcol}
    
  \end{tpdcol}\\

  \begin{tpacol}
    15:55
  \end{tpacol}&
  \begin{tpbcol}
    司会
  \end{tpbcol}&
  \begin{tpccol}
    班につく先生の紹介    
  \end{tpccol}&
  \begin{tpdcol}
    2班ずつ体育館から野外炊事場に移動するように伝える \par
    班付が一時的に別行動となるため,前の班に続いて移動するように伝える \par
    靴袋は明日も使うため無くさないように持っておくことを伝える
  \end{tpdcol}\\
  
  \begin{tpacol}
    16:00
  \end{tpacol}&
  \begin{tpbcol}
    誘導
  \end{tpbcol}&
  \begin{tpccol}
    班ごとに野外炊事場へ移動
  \end{tpccol}&
  \begin{tpdcol}
    列が長くなりすぎないように,適度な場所で列を切る
  \end{tpdcol}\\
  
\end{longtable}
\normalsize

\subsection{準備物品}
\begin{table}[h]
  \centering
  \caption{物品}
  \begin{tabular}{lrl}
    \hline
    \multicolumn{1}{c}{物品} & \multicolumn{1}{c}{数量} & \multicolumn{1}{c}{使用用途・備考} \\
    \hline
    マイク & 1 & 司会用 \\
    スピーカー & 1 & 司会用 \\
    \hline  
  \end{tabular}
\end{table}

\subsection{備考}
\begin{itemize}
  \item 班代表が揃っていない場合は、前後の班代表が整列を担当する
  \item 雨天時は靴袋と共に傘袋も配布する
\end{itemize}


\subsection{全体配置}

\begin{center}
    \begin{tikzpicture}
        % 部屋の外枠
        \draw (12,3) -- (12,7) -- (8.5,10.5) -- (1.5,10.5) -- (-2,7) -- (-2,3) -- cycle;
        
        % 出入口
        \draw (-2,4) rectangle (-1,6);
        \node at (-1.5,5.4) {\small 出};
        \node at (-1.5,5.1) {\small 入};
        \node at (-1.5,4.8) {\small 口};
        
        % 司会者席
        \draw (1,8.5) rectangle (1.75,9.25);
        \node at (1.4,8.9) {\small 司};
        
        % 教職員席(長方形)
        \draw (2,8.5) rectangle (8,10);
        \node at (5,9.235) {教職員の座席};
        
        % 左側の座席
        \foreach \x in {0.5, 1.5, 2.5} {
            \foreach \y in {7.5, 6.8, 6.1} {
                \draw[thick] (\x,\y) circle(0.3);
            }
        }

        \node at (0.5,8) {1};
        \node at (1.5,8) {2};
        \node at (2.5,8) {3};

        \foreach \x in {0.5, 1.5, 2.5} {
            \foreach \y in {5.5, 5.2, 4.9} {
                \fill (\x,\y) circle(0.05);
            }
        }

        \foreach \x in {0.5, 1.5, 2.5} {
            \foreach \y in {4.3} {
                \draw[thick] (\x,\y) circle(0.3);
            }
        }
        
        % 右側の座席
        \foreach \x in {7.5, 8.5, 9.5} {
            \foreach \y in {7.5, 6.8, 6.1} {
                \draw[thick] (\x,\y) circle(0.3);
            }
        }
        
        \foreach \x in {7.5, 8.5, 9.5} {
            \foreach \y in {5.5, 5.2, 4.9} {
                \fill (\x,\y) circle(0.05);
            }
        }

        \foreach \x in {7.5, 8.5, 9.5} {
            \foreach \y in {4.3} {
                \draw[thick] (\x,\y) circle(0.3);
            }
        }

        \node at (7.5,8) {18};
        \node at (8.5,8) {19};
        \node at (9.5,8) {20};
        
        % 中央の座席(点で表現)
        \foreach \x in {4, 5, 6} {
            \foreach \y in {7.5, 6.8, 6.1} {
                \fill (\x,\y) circle(0.05);
            }
        }

        \foreach \x in {4, 5, 6} {
            \foreach \y in {4.3} {
                \fill (\x,\y) circle(0.05);
            }
        }
        
    \end{tikzpicture}
\end{center}
