% ファイル名は「番号_タイトル.tex」
% 番号はNotionを参考
\section{サンプル}  % 割り振りの名前を記入

\subsection{場所・人員配置}
\begin{table}[h]
  \centering
  \begin{minipage}[t]{0.48\columnwidth}
    \centering
    \caption{場所・該当グループ}
    % グループは2/10に分けたやつです.企画書班のNotionにも追加しておくので参考にしておいてください.
    \begin{tabular}{cl}
      \hline
      場所 & 体育館 \\
      \hline
      グループ & 先遣隊 \\
       & 後遣隊 \\
      \hline
    \end{tabular}
  \end{minipage}
  \begin{minipage}[t]{0.48\columnwidth}
    \centering
    \caption{役割分担}
    \begin{tabular}{ll}
      \hline
      \multicolumn{1}{c}{役割} & \multicolumn{1}{c}{担当者} \\
      \hline
      役割1 & 人物A,人物B \\
      役割2 & 人物C,人物D,人物E \\
      \hline
    \end{tabular}
  \end{minipage}
\end{table}

\subsection{詳細タイムスケジュール}
\vspace{0.5em}
\fontsize{8pt}{12pt}\selectfont

\begin{longtable}[c]{|p{.04\textwidth}|p{.2\textwidth}|p{.25\textwidth}|p{.45\textwidth}|}
  \caption{スケジュール} \\
  \hline
  \multicolumn{1}{|c|}{時間} & \multicolumn{1}{c|}{担当} & \multicolumn{1}{c|}{内容} & \multicolumn{1}{c|}{備考} \\
  \hline
  \endfirsthead
  
  \hline
  \multicolumn{1}{|c|}{時間} & \multicolumn{1}{c|}{担当} & \multicolumn{1}{c|}{内容} & \multicolumn{1}{c|}{備考} \\
  \hline
  \endhead
  
  \hline
  \endfoot
  
  \hline
  \endlastfoot
  
  \begin{tpacol}
    09:00
    % 時間は4桁(〇〇:〇〇)で記入する
  \end{tpacol}&
  \begin{tpbcol}
    役割1
    % 役割or担当者を記入する
  \end{tpbcol}&
  \begin{tpccol}
    内容
  \end{tpccol}&
  \begin{tpdcol}
    備考(ない場合は空白でOK)
  \end{tpdcol}\\

  % 同じ時間でも担当が違う場合は別の行に記入する(その際の時間の記入の必要はなし)
  \begin{tpacol}
    
  \end{tpacol}&
  \begin{tpbcol}
    役割2
  \end{tpbcol}&
  \begin{tpccol}
    内容
  \end{tpccol}&
  \begin{tpdcol}
    備考 \par
    (表内改行可能です)
  \end{tpdcol}\\

  \begin{tpacol}
    
  \end{tpacol}&
  \begin{tpbcol}
    
  \end{tpbcol}&
  \begin{tpccol}
    
  \end{tpccol}&
  \begin{tpdcol}
    
  \end{tpdcol}\\


\end{longtable}
\normalsize

\subsection{準備物品}
\begin{table}[h]
  \centering
  \caption{物品}
  \begin{tabular}{lrl}
    \hline
    \multicolumn{1}{c}{物品} & \multicolumn{1}{c}{数量} & \multicolumn{1}{c}{使用用途・備考} \\
    \hline
    物品A & 2 & 使用用途・備考(特に記載する必要なければ空白でOK) \\
    物品B & 11 &  \\
    \hline  
  \end{tabular}
\end{table}

\subsection{備考}
% 特に書くことなければ省略してOK