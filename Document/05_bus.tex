\newpage

\section{バス内}  % 割り振りの名前を記入

\subsection{場所・人員配置}
\begin{table}[h]
  \centering
  \begin{minipage}[t]{0.48\columnwidth}
    \centering
    \caption{場所・該当グループ}
    \begin{tabular}{cl}
      \hline
      場所 & バス \\
      \hline
      グループ & 班付 \\
       & 裏方 \\
       & イベント運営 \\
      \hline
    \end{tabular}
  \end{minipage}
  \begin{minipage}[t]{0.48\columnwidth}
    \centering
    \caption{役割分担}
    \begin{tabular}{ll}
      \hline
      \multicolumn{1}{c}{役割} & \multicolumn{1}{c}{担当者} \\
      \hline
      司会(1号車) & 石本,納富 \\
      司会補助(1号車) & 猫田,公文 \\
      司会(2号車) & 松崎,黒木 \\
      司会補助(2号車) & 髙岡,加藤 \\
      司会(3号車) & 福澤,芳野 \\
      司会補助(3号車) & 打越,畠中 \\
      司会(4号車) & 大亦,細川 \\
      司会補助(4号車) & 小谷,田島 \\
      \hline
    \end{tabular}
  \end{minipage}
\end{table}

\subsection{詳細タイムスケジュール}
\vspace{0.5em}
\fontsize{8pt}{12pt}\selectfont

\begin{longtable}[c]{|p{.04\textwidth}|p{.2\textwidth}|p{.25\textwidth}|p{.45\textwidth}|}
  \caption{スケジュール} \\
  \hline
  \multicolumn{1}{|c|}{時間} & \multicolumn{1}{c|}{担当} & \multicolumn{1}{c|}{内容} & \multicolumn{1}{c|}{備考} \\
  \hline
  \endfirsthead
  
  \hline
  \multicolumn{1}{|c|}{時間} & \multicolumn{1}{c|}{担当} & \multicolumn{1}{c|}{内容} & \multicolumn{1}{c|}{備考} \\
  \hline
  \endhead
  
  \hline
  \endfoot
  
  \hline
  \endlastfoot
  
  \begin{tpacol}
    12:10
  \end{tpacol}&
  \begin{tpbcol}
    
  \end{tpbcol}&
  \begin{tpccol}
    バス出発
  \end{tpccol}&
  \begin{tpdcol}
    酔い止めセットがあるかもう一度確認する
  \end{tpdcol}\\

  \begin{tpacol}
    12:15
  \end{tpacol}&
  \begin{tpbcol}
    司会,司会補助
  \end{tpbcol}&
  \begin{tpccol}
    司会者挨拶,自己紹介\par
    流れ・諸注意説明
  \end{tpccol}&
  \begin{tpdcol}
    飲食は禁止,ゴミは自分で持っておく,気分が悪くなったときは近くのスタッフに声をかける旨の対応等を話す \par

  \end{tpdcol}\\

  \begin{tpacol}
    
  \end{tpacol}&
  \begin{tpbcol}
    
  \end{tpbcol}&
  \begin{tpccol}
    教員紹介
  \end{tpccol}&
  \begin{tpdcol}
    これ以降のバス内企画はイベント班企画書参考のこと
  \end{tpdcol}\\

  \begin{tpacol}
    12:55
  \end{tpacol}&
  \begin{tpbcol}
    
  \end{tpbcol}&
  \begin{tpccol}
    休憩場所到着5分前
  \end{tpccol}&
  \begin{tpdcol}
    もうすぐ休憩場所に到着する旨, \par
    出発時間5分前にバス内に着席することを説明する
  \end{tpdcol}\\

  \begin{tpacol}
    13:00
  \end{tpacol}&
  \begin{tpbcol}
    
  \end{tpbcol}&
  \begin{tpccol}
    休憩場所到着
  \end{tpccol}&
  \begin{tpdcol}
    休憩場所に到着した旨を\#当日報告にて連絡する
  \end{tpdcol}\\

  \begin{tpacol}
    13:15
  \end{tpacol}&
  \begin{tpbcol}
    
  \end{tpbcol}&
  \begin{tpccol}
    人数確認
  \end{tpccol}&
  \begin{tpdcol}
    気分が悪い人がいないか,隣の人がいるか確認し,人数を確認する
  \end{tpdcol}\\

  \begin{tpacol}
    13:20
  \end{tpacol}&
  \begin{tpbcol}
    
  \end{tpbcol}&
  \begin{tpccol}
    バス出発
  \end{tpccol}&
  \begin{tpdcol}
    バスが出発した旨を\#当日報告にて連絡する
  \end{tpdcol}\\

  \begin{tpacol}
    13:25
  \end{tpacol}&
  \begin{tpbcol}
    
  \end{tpbcol}&
  \begin{tpccol}
    流れ説明
  \end{tpccol}&
  \begin{tpdcol}
    到着予定時間と到着後の流れを説明する \par
    その後またバス内企画を行う
  \end{tpdcol}\\

  \begin{tpacol}
    14:35
  \end{tpacol}&
  \begin{tpbcol}
    
  \end{tpbcol}&
  \begin{tpccol}
    室戸到着5分前
  \end{tpccol}&
  \begin{tpdcol}
    山道に入ったあたりでもうすぐつく旨を\#当日報告にて連絡する
  \end{tpdcol}\\

  \begin{tpacol}
    14:40
  \end{tpacol}&
  \begin{tpbcol}
    
  \end{tpbcol}&
  \begin{tpccol}
    室戸到着
  \end{tpccol}&
  \begin{tpdcol}
    バスは前から順に速やかに降りてもらい,司会が誘導する \par
    到着した旨を\#当日報告に連絡する
  \end{tpdcol}\\


\end{longtable}
\normalsize

% \subsection{備考}
% 全員の体調確認を忘れない