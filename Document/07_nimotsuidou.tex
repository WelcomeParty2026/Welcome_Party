\newpage

\section{荷物移動}  % 割り振りの名前を記入

\subsection{場所・人員配置}
\begin{table}[h]
  \centering
  \begin{minipage}[t]{0.48\columnwidth}
    \centering
    \caption{場所・該当グループ}
    \begin{tabular}{cl}
      \hline
      場所 & プレイルーム \\
       & 宿泊棟 \\
      \hline
      グループ & 裏方 \\
       & イベント運営 \\
      \hline
    \end{tabular}
  \end{minipage}
  \begin{minipage}[t]{0.48\columnwidth}
    \centering
    \caption{役割分担}
    \begin{tabular}{ll}
      \hline
      \multicolumn{1}{c}{役割} & \multicolumn{1}{c}{担当者} \\
      \hline
      野外炊事準備 & 人物A,人物B,人物C,人物D\\
       & 人物E,人物F,人物G,人物H \\
      シーツ運び & 人物A,人物B,人物C\\
       & 人物D,人物E,人物F,人物G \\
       & 人物H,人物I \\
      シーツ運び総括 & 人物A \\
      延長ケーブル設置 & 人物A,人物B \\
      人物C & 人物D \\
      \hline
    \end{tabular}
  \end{minipage}
\end{table}

\subsection{詳細タイムスケジュール}
\vspace{0.5em}
\fontsize{8pt}{12pt}\selectfont

\begin{longtable}[c]{|p{.04\textwidth}|p{.2\textwidth}|p{.25\textwidth}|p{.45\textwidth}|}
  \caption{スケジュール} \\
  \hline
  \multicolumn{1}{|c|}{時間} & \multicolumn{1}{c|}{担当} & \multicolumn{1}{c|}{内容} & \multicolumn{1}{c|}{備考} \\
  \hline
  \endfirsthead
  
  \hline
  \multicolumn{1}{|c|}{時間} & \multicolumn{1}{c|}{担当} & \multicolumn{1}{c|}{内容} & \multicolumn{1}{c|}{備考} \\
  \hline
  \endhead
  
  \hline
  \endfoot
  
  \hline
  \endlastfoot
  
  \begin{tpacol}
    14:40
  \end{tpacol}&
  \begin{tpbcol}
    全員
  \end{tpbcol}&
  \begin{tpccol}
    荷物の積み下ろし
  \end{tpccol}&
  \begin{tpdcol}
    自分の荷物,バスに乗せてある物品を全てプレイルームに運ぶ
  \end{tpdcol}\\

  \begin{tpacol}
    15:00
  \end{tpacol}&
  \begin{tpbcol}
    野外炊事準備
  \end{tpbcol}&
  \begin{tpccol}
    物品準備
  \end{tpccol}&
  \begin{tpdcol}
    各班ごとに分けて準備する \par
    当日に変更がある可能性があるため,臨機応変に動く \par
    野外炊事打ち合わせメンバー,施設職員の指示に従う \par
    新入生が来た場合は,どこが何班か案内する
  \end{tpdcol}\\

  \begin{tpacol}
    
  \end{tpacol}&
  \begin{tpbcol}
    延長ケーブル設置
  \end{tpbcol}&
  \begin{tpccol}
    延長ケーブル設置
  \end{tpccol}&
  \begin{tpdcol}
    各宿泊棟に延長コードを分配する \par
    終わりしだシーツ運びに合流する
  \end{tpdcol}\\

  \begin{tpacol}
    
  \end{tpacol}&
  \begin{tpbcol}
    シーツ運び
  \end{tpbcol}&
  \begin{tpccol}
    シーツ運び
  \end{tpccol}&
  \begin{tpdcol}
    工科大としてまとめられているシーツを一旦全てA棟に運び,そこから数を数えて分配する \par
    慎太郎の分は別でわかるようにしておく \par
    終わった時間によって,入所式に途中参加するか,そのまま野外炊事に行くか,臨機応変に動く \par
    総括の指示に従う
  \end{tpdcol}\\

  \begin{tpacol}
    16:00
  \end{tpacol}&
  \begin{tpbcol}
    全員
  \end{tpbcol}&
  \begin{tpccol}
    野外炊事場に移動   
  \end{tpccol}&
  \begin{tpdcol}
    
  \end{tpdcol}\\

  \begin{tpacol}
    
  \end{tpacol}&
  \begin{tpbcol}
    総括,鍵係
  \end{tpbcol}&
  \begin{tpccol}
    鍵閉め
  \end{tpccol}&
  \begin{tpdcol}
    シーツの移動が完了し次第鍵係に連絡し,宿泊棟の鍵を閉める
  \end{tpdcol}\\


\end{longtable}
\normalsize

\newpage

\subsection{準備物品}
\begin{table}[h]
  \centering
  \caption{物品}
  \begin{tabular}{lrl}
    \hline
    \multicolumn{1}{c}{物品} & \multicolumn{1}{c}{数量} & \multicolumn{1}{c}{使用用途・備考} \\
    \hline
    軍手 & 40 & 各班2つ \\
    ふきん & 60 & 各班3つ \\
    台ふきん & 40 & 各班2つ \\
    ペーパータオル & 4 & ブロックで1つ \\
    新聞紙 & 20 & 各班1枚 \\
    \hline  
  \end{tabular}
\end{table}
