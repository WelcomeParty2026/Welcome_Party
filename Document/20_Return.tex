\newpage

\section{先遣隊・後遣隊}  % 割り振りの名前を記入

\subsection{場所・人員配置}
\begin{table}[h]
  \centering
  \begin{minipage}[t]{0.48\columnwidth}
    \centering
    \caption{場所・該当グループ}
    \begin{tabular}{cl}
      \hline
      場所 & バス \\
      \hline
      グループ & バスに乗るスタッフ \\
      \hline
    \end{tabular}
  \end{minipage}
  \begin{minipage}[t]{0.48\columnwidth}
    \centering
    \caption{役割分担}
    \begin{tabular}{ll}
      \hline
      \multicolumn{1}{c}{役割} & \multicolumn{1}{c}{担当者} \\
      \hline
      1号車 & 人物A,人物B \\
      2号車 & 人物C,人物D \\
      3号車 & 人物E, 人物F \\
      \hline
    \end{tabular}
  \end{minipage}
\end{table}

\subsection{詳細タイムスケジュール}
\vspace{0.5em}
\fontsize{8pt}{12pt}\selectfont

\begin{longtable}[c]{|p{.04\textwidth}|p{.2\textwidth}|p{.25\textwidth}|p{.45\textwidth}|}
  \caption{スケジュール} \\
  \hline
  \multicolumn{1}{|c|}{時間} & \multicolumn{1}{c|}{担当} & \multicolumn{1}{c|}{内容} & \multicolumn{1}{c|}{備考} \\
  \hline
  \endfirsthead
  
  \hline
  \multicolumn{1}{|c|}{時間} & \multicolumn{1}{c|}{担当} & \multicolumn{1}{c|}{内容} & \multicolumn{1}{c|}{備考} \\
  \hline
  \endhead
  
  \hline
  \endfoot
  
  \hline
  \endlastfoot
  
  \begin{tpacol}
    13:00
  \end{tpacol}&
  \begin{tpbcol}
    司会者
  \end{tpbcol}&
  \begin{tpccol}
    室戸少年自然の家出発 \par 新入生に感想などを聞いてみる.
  \end{tpccol}&
  \begin{tpdcol}
    疲れていそうならば控える.
  \end{tpdcol}\\

  
  \begin{tpacol}
    14:15  
  \end{tpacol}&
  \begin{tpbcol}
    司会者
  \end{tpbcol}&
  \begin{tpccol}
    安芸駅(安芸球場)到着5分前 \par トイレについての説明, 5分前に集合することを話す.
  \end{tpccol}&
  \begin{tpdcol}

  \end{tpdcol}\\

  \begin{tpacol}
    14:20
  \end{tpacol}&
  \begin{tpbcol}
    司会者
  \end{tpbcol}&
  \begin{tpccol}
    安芸駅(安芸球場)到着 \par 休憩時間を伝える. 到着の旨をSlackに連絡する.
  \end{tpccol}&
  \begin{tpdcol}
    休憩時間約20分. 
  \end{tpdcol}\\

  \begin{tpacol}
    14:35
  \end{tpacol}&
  \begin{tpbcol}
    司会者
  \end{tpbcol}&
  \begin{tpccol}
    安芸駅(安芸球場)出発5分前 \par 司会者が人数チェックする.
  \end{tpccol}&
  \begin{tpdcol}
    
  \end{tpdcol}\\

  \begin{tpacol}
    14:40
  \end{tpacol}&
  \begin{tpbcol}
    司会者
  \end{tpbcol}&
  \begin{tpccol}
    安芸駅(安芸球場)出発 \par 工科大到着時刻と到着後各自解散することを伝える.
  \end{tpccol}&
  \begin{tpdcol}
    出発後のバスの時間はフリーな時間とする.
  \end{tpdcol}\\

  \begin{tpacol}
    15:55
  \end{tpacol}&
  \begin{tpbcol}
    司会者
  \end{tpbcol}&
  \begin{tpccol}
    高知工科大学到着5分前 \par 荷物を持ち, 流れ解散であることを伝える.
  \end{tpccol}&
  \begin{tpdcol}
    降車の際, 名前を回収する事を忘れず伝える.
  \end{tpdcol}\\

  \begin{tpacol}
    16:00 
  \end{tpacol}&
  \begin{tpbcol}
    司会者
  \end{tpbcol}&
  \begin{tpccol}
    高知工科大学到着 \par 到着の旨をSlackに連絡する. 最初に補助席に座っているスタッフが降り, 名札を回収する.
  \end{tpccol}&
  \begin{tpdcol}
    手の空いているスタッフが荷物を降ろす.
  \end{tpdcol}\\

\end{longtable}
\normalsize

\subsection{準備物品}
\begin{table}[h]
  \centering
  \caption{物品}
  \begin{tabular}{lrl}
    \hline
    \multicolumn{1}{c}{物品} & \multicolumn{1}{c}{数量} & \multicolumn{1}{c}{使用用途・備考} \\
    \hline
    酔い止め薬 & 各バス1箱 &  \\
    エチケット袋 & 各バス2枚 &  \\
    紙コップ & 各バス5個 & \\
    水 & 各バス500ml1本 & 水は常温 \\
    名札回収用袋 & 3つ & \\
    \hline  
  \end{tabular}
\end{table}

\subsection{備考}
% 特に書くことなければ省略してOK