\newpage

\section{消灯}

\subsection{場所・人員配置}
\begin{table}[h]
  \centering
  \begin{minipage}[t]{0.48\columnwidth}
    \centering
    \caption{場所・該当グループ}
    \begin{tabular}{cl}
      \hline
      場所 & 体育館 \\
      \hline
      グループ & 先遣隊 \\
       & 後遣隊 \\
      \hline
    \end{tabular}
  \end{minipage}
  \begin{minipage}[t]{0.48\columnwidth}
    \centering
    \caption{役割分担}
    \begin{tabular}{ll}
      \hline
      \multicolumn{1}{c}{役割} & \multicolumn{1}{c}{担当者} \\
      \hline
      役割1 & 人物A,人物B \\
      役割2 & 人物C,人物D,人物E \\
      \hline
    \end{tabular}
  \end{minipage}
\end{table}

\subsection{詳細タイムスケジュール}
\vspace{0.5em}
\fontsize{8pt}{12pt}\selectfont

\begin{longtable}[c]{|p{.04\textwidth}|p{.2\textwidth}|p{.25\textwidth}|p{.45\textwidth}|}
  \caption{スケジュール} \\
  \hline
  \multicolumn{1}{|c|}{時間} & \multicolumn{1}{c|}{担当} & \multicolumn{1}{c|}{内容} & \multicolumn{1}{c|}{備考} \\
  \hline
  \endfirsthead
  
  \hline
  \multicolumn{1}{|c|}{時間} & \multicolumn{1}{c|}{担当} & \multicolumn{1}{c|}{内容} & \multicolumn{1}{c|}{備考} \\
  \hline
  \endhead
  
  \hline
  \endfoot
  
  \hline
  \endlastfoot
  
  \begin{tpacol}
    09:00
    % 時間は4桁(〇〇:〇〇)で記入する
  \end{tpacol}&
  \begin{tpbcol}
    役割1
    % 役割or担当者を記入する
  \end{tpbcol}&
  \begin{tpccol}
    内容
  \end{tpccol}&
  \begin{tpdcol}
    備考(ない場合は空白でOK)
  \end{tpdcol}\\

  % 同じ時間でも担当が違う場合は別の行に記入する(その際の時間の記入の必要はなし)
  \begin{tpacol}
    
  \end{tpacol}&
  \begin{tpbcol}
    役割2
  \end{tpbcol}&
  \begin{tpccol}
    内容
  \end{tpccol}&
  \begin{tpdcol}
    備考 \par
    (表内改行可能です)
  \end{tpdcol}\\

  \begin{tpacol}
    
  \end{tpacol}&
  \begin{tpbcol}
    
  \end{tpbcol}&
  \begin{tpccol}
    
  \end{tpccol}&
  \begin{tpdcol}
    
  \end{tpdcol}\\


\end{longtable}
\normalsize

\subsection{準備物品}
\begin{table}[h]
  \centering
  \caption{物品}
  \begin{tabular}{lrl}
    \hline
    \multicolumn{1}{c}{物品} & \multicolumn{1}{c}{数量} & \multicolumn{1}{c}{使用用途・備考} \\
    \hline
    物品A & 2 & 使用用途・備考(特に記載する必要なければ空白でOK) \\
    物品B & 11 &  \\
    \hline  
  \end{tabular}
\end{table}

\subsection{備考}




\subsection{日時・場所}
\begin{description}
  \item[日時 :] 2025年4月6日(日) 19:00 $\sim$ 20:00
  \item[場所 :] 研修室2,体育館
\end{description}

\subsection{目的}
野外炊事が終わり,担当グループの新入生を多目的ホールまで誘導後,研修室2に集合する.
%人数3
  は機材(パソコン,カメラ等)の動作確認をする.必要であれば机や椅子などを体育館に配置する.
イベントで使う備品は研修室2にまとめて置いているので,そこからプラカードと先生方用の椅子(15脚)を配置する.
タイムスケジュールに記載しているように,20:00まで作業可能とする.作業が早く終了したらその時点で終了とする.鍵の開け閉めを  が行う.現場は  の指揮で動く.

\subsection{タイムスケジュール}
%人数2
\begin{description}
  \item[19:00] \textbf{スタッフ集合・作業開始}
  \begin{itemize}
    \item   が研修室2の鍵を借り,解錠する
    \item   がSlackの\#当日報告に連絡する
    \item 研修室2に集合し次第,作業開始
  \end{itemize}

  \item[20:00] \textbf{作業終了}
  \begin{itemize}
    \item 最長作業終了時間
  \end{itemize}
\end{description}

\subsection{人員配置}
%人数2
鍵係:   \par
イベント統括 :   

\subsection{必要物品}
\begin{itemize}
  \item 机:1台
  \item 椅子:15脚
  \item プロジェクタ:1台
  \item スクリーン:1枚
  \item PC:1台
  \item カメラ(iphoneを使用)
  \item 延長ケーブル
  \item その他イベントで使用する予定の備品
\end{itemize}

\subsection{備考}
%人数1
\begin{itemize}
  \item 机,椅子は体育館にあるものを使用する
  \item マイクの電池を確認しておく
  \item 鍵を借り忘れたことがあるそうなので,鍵係は忘れずに鍵を借りる
  \item   はすぐ連絡が取れるようにスマホを常備しておく
\end{itemize}

\subsection{日時・場所}
\begin{description}
  \item[日時 :] 2025年4月6日(日)  ??:?? $\sim$
  \item[場所 :] 宿泊棟の休憩室
\end{description}

\subsection{目的}
翌日のイベントをスムーズに行なうため.

\subsection{タイムスケジュール}
\begin{description}
  \item[22:00] \textbf{休憩室集合}
  \begin{itemize}
    \item 打ち合わせに参加するメンバーは22:00に参加できるように行動する
    \item 宴に参加している場合は飲み過ぎないように注意する
  \end{itemize}

  \item[22:10] \textbf{打ち合わせ開始}
  \begin{itemize}
    \item 1日目の反省,気になる点などの確認・報告をする
    \item 翌日のイベントを中心に,一度流れを確認する
    \item 新入生の様子,先生方の様子などで気になることがあれば必ず報告する
    \item イベント準備を行うスタッフ(9名)は朝の集いのタイムスケジュールを確認しつつ,朝の集い中に朝食へ抜けるタイミングを決める
  \end{itemize}

  \item[23:00] \textbf{打ち合わせ終了}
\end{description}

\subsection{人員配置}
%人数9


\subsection{備考}
\begin{itemize}
  \item 宴に参加しているスタッフは集合に間に合うように注意する
  \item 前日準備の進捗を確認する
\end{itemize}