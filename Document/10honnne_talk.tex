\newpage

\section{本音トーク}

\subsection{場所・人員配置}
\begin{table}[h]
  \centering
  \begin{minipage}[t]{0.48\columnwidth}
    \centering
    \caption{場所・該当グループ}
    \begin{tabular}{cl}
      \hline
      場所 & 多目的ホール \\
      \hline
      グループ & 先遣隊 \\
       & 後遣隊 \\
      \hline
    \end{tabular}
  \end{minipage}
  \begin{minipage}[t]{0.48\columnwidth}
    \centering
    \caption{役割分担}
    \begin{tabular}{ll}
      \hline
      \multicolumn{1}{c}{役割} & \multicolumn{1}{c}{担当者} \\
      \hline
      お菓子配置 & 芦田,佐藤謙,佐藤央 \\
      飲み物準備 & 眞鍋,森田,仲村,高橋 \\
      \hline
    \end{tabular}
  \end{minipage}
\end{table}

\subsection{詳細タイムスケジュール}
\vspace{0.5em}
\fontsize{8pt}{12pt}\selectfont

\begin{longtable}[c]{|p{.04\textwidth}|p{.2\textwidth}|p{.25\textwidth}|p{.45\textwidth}|}
  \caption{スケジュール} \\
  \hline
  \multicolumn{1}{|c|}{時間} & \multicolumn{1}{c|}{担当} & \multicolumn{1}{c|}{内容} & \multicolumn{1}{c|}{備考} \\
  \hline
  \endfirsthead
  
  \hline
  \multicolumn{1}{|c|}{時間} & \multicolumn{1}{c|}{担当} & \multicolumn{1}{c|}{内容} & \multicolumn{1}{c|}{備考} \\
  \hline
  \endhead
  
  \hline
  \endfoot
  
  \hline
  \endlastfoot
  
  \begin{tpacol}
    19:00
  \end{tpacol}&
  \begin{tpbcol}
    全員
  \end{tpbcol}&
  \begin{tpccol}
    本音トーク準備
  \end{tpccol}&
  \begin{tpdcol}
    お菓子とジュースを準備する \par
  \end{tpdcol}\\

  \begin{tpacol}
    19:30
  \end{tpacol}&
  \begin{tpbcol}
    全員
  \end{tpbcol}&
  \begin{tpccol}
    誘導
  \end{tpccol}&
  \begin{tpdcol}
    お風呂に入った人から誘導する
  \end{tpdcol}\\

  \begin{tpacol}
    21:10
  \end{tpacol}&
  \begin{tpbcol}
    芦田
  \end{tpbcol}&
  \begin{tpccol}
    忘れ物アナウンス
  \end{tpccol}&
  \begin{tpdcol}
    浴槽にあった忘れ物をアナウンスする
  \end{tpdcol}\\

  \begin{tpacol}
    21:45
  \end{tpacol}&
  \begin{tpbcol}
    全員
  \end{tpbcol}&
  \begin{tpccol}
    新入生への呼びかけ・撤収作業
  \end{tpccol}&
  \begin{tpdcol}
    ゴミを分別し,各宿泊棟の入り口付近にまとめて置いておく \par
    余った飲み物と食べ物はA棟に運ぶ \par
    翌日の起床時間,起きてからのタイムスケジュールを軽く説明し,寝るよう促す \par
    22時に消灯見回りに行くことやこれ以降の棟の出入りを禁止することも伝える \par
  \end{tpdcol}\\


\end{longtable}
\normalsize

\subsection{準備物品}
\begin{table}[h]
  \centering
  \caption{物品}
  \begin{tabular}{lr}
    \hline
    \multicolumn{1}{c}{物品} & \multicolumn{1}{c}{数量}  \\
    \hline
    お菓子 & たくさん \\
    ジュース & たくさん  \\
    \hline  
  \end{tabular}
\end{table}

\subsection{備考}
手の空いているスタッフは積極的に手伝う \par
先生方も参加される可能性がある \par
お酒を飲んだスタッフは行かない \par