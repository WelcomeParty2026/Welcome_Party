\newpage

\section{先遣隊の動き(室戸到着後)} 

\subsection{場所・人員配置}
\begin{table}[h]
  \centering
  \begin{minipage}[t]{0.48\columnwidth}
    \centering
    \caption{場所・該当グループ}
    \begin{tabular}{cl}
      \hline
      場所 & 事務室 \\
       & 体育館 \\
       & プレイルーム \\
      \hline
      グループ & 先遣隊 \\
      \hline
    \end{tabular}
  \end{minipage}
  \begin{minipage}[t]{0.48\columnwidth}
    \centering
    \caption{役割分担}
    \begin{tabular}{ll}
      \hline
      \multicolumn{1}{c}{役割} & \multicolumn{1}{c}{担当者} \\
      \hline
      全体打ち合わせ & 妻鳥先生,芦田 \\
      体育館設営 & 佐藤謙,佐藤央 \\
      プレイルーム設営 & 眞鍋,森田 \\
      野外炊事打ち合わせ & 佐藤謙,佐藤央 \\
      鍵係 & 眞鍋 \\
      \hline
    \end{tabular}
  \end{minipage}
\end{table}

\subsection{詳細タイムスケジュール}
\vspace{0.5em}
\fontsize{8pt}{12pt}\selectfont

\begin{longtable}[c]{|p{.04\textwidth}|p{.2\textwidth}|p{.25\textwidth}|p{.45\textwidth}|}
  \caption{スケジュール} \\
  \hline
  \multicolumn{1}{|c|}{時間} & \multicolumn{1}{c|}{担当} & \multicolumn{1}{c|}{内容} & \multicolumn{1}{c|}{備考} \\
  \hline
  \endfirsthead
  
  \hline
  \multicolumn{1}{|c|}{時間} & \multicolumn{1}{c|}{担当} & \multicolumn{1}{c|}{内容} & \multicolumn{1}{c|}{備考} \\
  \hline
  \endhead
  
  \hline
  \endfoot
  
  \hline
  \endlastfoot
  
  \begin{tpacol}
    13:00
  \end{tpacol}&
  \begin{tpbcol}
    全員
  \end{tpbcol}&
  \begin{tpccol}
    室戸到着
  \end{tpccol}&
  \begin{tpdcol}
    到着した旨を芦田が\#当日報告に連絡する
  \end{tpdcol}\\

  \begin{tpacol}
    13:10
  \end{tpacol}&
  \begin{tpbcol}
    芦田,妻鳥先生
  \end{tpbcol}&
  \begin{tpccol}
    室戸の方と打ち合わせ
  \end{tpccol}&
  \begin{tpdcol}
    打ち合わせの前に利用する部屋の設営の許可と鍵を頂き,眞鍋に渡す \par
    15時15分ごろに室戸職員の方に入所式に来ていただきたい旨を伝える
  \end{tpdcol}\\

  \begin{tpacol}
    
  \end{tpacol}&
  \begin{tpbcol}
    佐藤謙,佐藤央
  \end{tpbcol}&
  \begin{tpccol}
    体育館設営
  \end{tpccol}&
  \begin{tpdcol}
    教員用の椅子とマイク,プラカードを準備する(全体配置参考) \par
    動作確認も行う \par
    教員名を印刷した紙を教員用の椅子の背に貼る
  \end{tpdcol}\\

  \begin{tpacol}
    
  \end{tpacol}&
  \begin{tpbcol}
    眞鍋,森田
  \end{tpbcol}&
  \begin{tpccol}
    プレイルーム設営
  \end{tpccol}&
  \begin{tpdcol}
    野外炊事の班ごとのプラカードを配置する \par
    先生方の荷物は後ろに置くように配置する \par
    終わり次第体育館設営に合流する
  \end{tpdcol}\\

  \begin{tpacol}
    14:10
  \end{tpacol}&
  \begin{tpbcol}
    全員
  \end{tpbcol}&
  \begin{tpccol}
    荷物搬入
  \end{tpccol}&
  \begin{tpdcol}
    車に乗っている荷物を全て下ろす \par
    本隊の到着予定状況によって臨機応変に動く
  \end{tpdcol}\\

  \begin{tpacol}
    14:40
  \end{tpacol}&
  \begin{tpbcol}
    眞鍋,森田,佐藤謙,佐藤央
  \end{tpbcol}&
  \begin{tpccol}
    靴袋の配布
  \end{tpccol}&
  \begin{tpdcol}
    体育館入り口にて靴袋を配布する
  \end{tpdcol}\\

  \begin{tpacol}
    15:00
  \end{tpacol}&
  \begin{tpbcol}
    佐藤謙,佐藤央
  \end{tpbcol}&
  \begin{tpccol}
    野外炊事打ち合わせ
  \end{tpccol}&
  \begin{tpdcol}
    
  \end{tpdcol}\\


\end{longtable}
\normalsize

\newpage

\subsection{準備物品}
\begin{table}[h]
  \centering
  \caption{物品}
  \begin{tabular}{lrl}
    \hline
    \multicolumn{1}{c}{物品} & \multicolumn{1}{c}{数量} & \multicolumn{1}{c}{使用用途・備考} \\
    \hline
    マイク & 1 & 体育館で使用,室戸で借用 \\
    スピーカー & 1 & 体育館で使用,室戸で借用 \\
    プラカード & 2セット & 体育館,プレイルームで1セットずつ使用 \\
    教員名を印刷した紙 & 1セット & 体育館で使用 \\
    白ガムテ & 1 & 椅子に教職員名を貼り付ける \\
    靴袋 & 200 &  \\
    \hline  
  \end{tabular}
\end{table}

\subsection{体育館配置}

\begin{center}
    \begin{tikzpicture}
        % 部屋の外枠
        \draw (12,3) -- (12,7) -- (8.5,10.5) -- (1.5,10.5) -- (-2,7) -- (-2,3) -- cycle;
        
        % 出入口
        \draw (-2,4) rectangle (-1,6);
        \node at (-1.5,5.4) {\small 出};
        \node at (-1.5,5.1) {\small 入};
        \node at (-1.5,4.8) {\small 口};
        
        % 司会者席
        \draw (1,8.5) rectangle (1.75,9.25);
        \node at (1.4,8.9) {\small 司};
        
        % 教職員席(長方形)
        \draw (2,8.5) rectangle (8,10);
        \node at (5,9.235) {教職員の座席};
        
        % 左側の座席
        \foreach \x in {0.5, 1.5, 2.5} {
            \foreach \y in {7.5, 6.8, 6.1} {
                \draw[thick] (\x,\y) circle(0.3);
            }
        }

        \node at (0.5,8) {1};
        \node at (1.5,8) {2};
        \node at (2.5,8) {3};

        \foreach \x in {0.5, 1.5, 2.5} {
            \foreach \y in {5.5, 5.2, 4.9} {
                \fill (\x,\y) circle(0.05);
            }
        }

        \foreach \x in {0.5, 1.5, 2.5} {
            \foreach \y in {4.3} {
                \draw[thick] (\x,\y) circle(0.3);
            }
        }
        
        % 右側の座席
        \foreach \x in {7.5, 8.5, 9.5} {
            \foreach \y in {7.5, 6.8, 6.1} {
                \draw[thick] (\x,\y) circle(0.3);
            }
        }
        
        \foreach \x in {7.5, 8.5, 9.5} {
            \foreach \y in {5.5, 5.2, 4.9} {
                \fill (\x,\y) circle(0.05);
            }
        }

        \foreach \x in {7.5, 8.5, 9.5} {
            \foreach \y in {4.3} {
                \draw[thick] (\x,\y) circle(0.3);
            }
        }

        \node at (7.5,8) {18};
        \node at (8.5,8) {19};
        \node at (9.5,8) {20};
        
        % 中央の座席(点で表現)
        \foreach \x in {4, 5, 6} {
            \foreach \y in {7.5, 6.8, 6.1} {
                \fill (\x,\y) circle(0.05);
            }
        }

        \foreach \x in {4, 5, 6} {
            \foreach \y in {4.3} {
                \fill (\x,\y) circle(0.05);
            }
        }
        
    \end{tikzpicture}
\end{center}
