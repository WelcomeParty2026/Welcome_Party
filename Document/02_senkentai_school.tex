\newpage

\section{先遣隊の動き(学校)} 

\subsection{場所・人員配置}
\begin{table}[h]
  \centering
  \begin{minipage}[t]{0.48\columnwidth}
    \centering
    \caption{場所・該当グループ}
    \begin{tabular}{cl}
      \hline
      場所 & A棟(A355) \\
      \hline
      グループ & 先遣隊 \\
      \hline
    \end{tabular}
  \end{minipage}
  \begin{minipage}[t]{0.48\columnwidth}
    \centering
    \caption{役割分担}
    \begin{tabular}{ll}
      \hline
      \multicolumn{1}{c}{役割} & \multicolumn{1}{c}{担当者} \\
      \hline
      先遣隊1 & 妻鳥先生(車),芦田 \\
      先遣隊2 & 佐藤謙(車),佐藤央 \\
      先遣隊3 & 眞鍋(車),森田 \\
      \hline
    \end{tabular}
  \end{minipage}
\end{table}

\subsection{詳細タイムスケジュール}
\vspace{0.5em}
\fontsize{8pt}{12pt}\selectfont

\begin{longtable}[c]{|p{.04\textwidth}|p{.2\textwidth}|p{.25\textwidth}|p{.45\textwidth}|}
  \caption{スケジュール} \\
  \hline
  \multicolumn{1}{|c|}{時間} & \multicolumn{1}{c|}{担当} & \multicolumn{1}{c|}{内容} & \multicolumn{1}{c|}{備考} \\
  \hline
  \endfirsthead
  
  \hline
  \multicolumn{1}{|c|}{時間} & \multicolumn{1}{c|}{担当} & \multicolumn{1}{c|}{内容} & \multicolumn{1}{c|}{備考} \\
  \hline
  \endhead
  
  \hline
  \endfoot
  
  \hline
  \endlastfoot
  
  \begin{tpacol}
    10:15
  \end{tpacol}&
  \begin{tpbcol}
    佐藤謙,眞鍋,妻鳥先生
  \end{tpbcol}&
  \begin{tpccol}
    車をA棟西側に移動させる
  \end{tpccol}&
  \begin{tpdcol}
    佐藤謙,眞鍋の車は事前に車両申請しておく
  \end{tpdcol}\\

  \begin{tpacol}
    
  \end{tpacol}&
  \begin{tpbcol}
    芦田,佐藤央,森田
  \end{tpbcol}&
  \begin{tpccol}
    荷物を移動させる
  \end{tpccol}&
  \begin{tpdcol}
    荷物はA355にまとめておく
  \end{tpdcol}\\

  \begin{tpacol}
    10:50
  \end{tpacol}&
  \begin{tpbcol}
    全員
  \end{tpbcol}&
  \begin{tpccol}
    荷物の最終確認
  \end{tpccol}&
  \begin{tpdcol}
    チェックリストと照らし合わせ,忘れ物がないかを確認する
  \end{tpdcol}\\

  \begin{tpacol}
    11:00
  \end{tpacol}&
  \begin{tpbcol}
    全員
  \end{tpbcol}&
  \begin{tpccol}
    工科大出発
  \end{tpccol}&
  \begin{tpdcol}
    途中で休憩をとりながら室戸に向かう
  \end{tpdcol}\\

\end{longtable}
\normalsize

\subsection{準備物品}
\begin{table}[h]
  \centering
  \caption{物品}
  \begin{tabular}{lrl}
    \hline
    \multicolumn{1}{c}{物品} & \multicolumn{1}{c}{数量} & \multicolumn{1}{c}{使用用途・備考} \\
    \hline
    チェックリスト & 1 & 物品の内訳とどの車に乗せたかを明確にしておく \\
    \hline  
  \end{tabular}
\end{table}