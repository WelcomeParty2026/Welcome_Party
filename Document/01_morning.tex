<<<<<<< HEAD
\section{朝の動き}
=======
% ファイル名は「番号_タイトル.tex」
% 番号はNotionを参考
\section{朝の動き}  % 割り振りの名前を記入
>>>>>>> 0dc73cf63a5cf1767d6c1c4abd7880ac2b8de408

\subsection{場所・人員配置}
\begin{table}[h]
  \centering
  \begin{minipage}[t]{0.48\columnwidth}
    \centering
    \caption{場所・該当グループ}
<<<<<<< HEAD
    \begin{tabular}{cl}
      \hline
      場所 & K101 \\
      \hline
      グループ & 全員 \\
=======
    % グループは2/10に分けたやつです.企画書班のNotionにも追加しておくので参考にしておいてください.
    \begin{tabular}{cl}
      \hline
      場所 & 体育館 \\
      \hline
      グループ & 先遣隊 \\
       & 後遣隊 \\
>>>>>>> 0dc73cf63a5cf1767d6c1c4abd7880ac2b8de408
      \hline
    \end{tabular}
  \end{minipage}
  \begin{minipage}[t]{0.48\columnwidth}
    \centering
    \caption{役割分担}
    \begin{tabular}{ll}
      \hline
      \multicolumn{1}{c}{役割} & \multicolumn{1}{c}{担当者} \\
      \hline
<<<<<<< HEAD
      説明 & 芦田 \\
      出席確認 & 花井,黒木 \\
      集金 & 野本,楠本 \\
=======
      役割1 & 人物A,人物B \\
      役割2 & 人物C,人物D,人物E \\
>>>>>>> 0dc73cf63a5cf1767d6c1c4abd7880ac2b8de408
      \hline
    \end{tabular}
  \end{minipage}
\end{table}

\subsection{詳細タイムスケジュール}
\vspace{0.5em}
\fontsize{8pt}{12pt}\selectfont

\begin{longtable}[c]{|p{.04\textwidth}|p{.2\textwidth}|p{.25\textwidth}|p{.45\textwidth}|}
  \caption{スケジュール} \\
  \hline
  \multicolumn{1}{|c|}{時間} & \multicolumn{1}{c|}{担当} & \multicolumn{1}{c|}{内容} & \multicolumn{1}{c|}{備考} \\
  \hline
  \endfirsthead
  
  \hline
  \multicolumn{1}{|c|}{時間} & \multicolumn{1}{c|}{担当} & \multicolumn{1}{c|}{内容} & \multicolumn{1}{c|}{備考} \\
  \hline
  \endhead
  
  \hline
  \endfoot
  
  \hline
  \endlastfoot
  
  \begin{tpacol}
<<<<<<< HEAD
    06:00
  \end{tpacol}&
  \begin{tpbcol}
    全員
  \end{tpbcol}&
  \begin{tpccol}
    各自起床
  \end{tpccol}&
  \begin{tpdcol}
    起床したらslackの\#当日報告に起きた旨を連絡する\par
    連絡がない場合は電話する
  \end{tpdcol}\\

  \begin{tpacol}
    07:00
  \end{tpacol}&
  \begin{tpbcol}
    全員 \par
    ~~~出席確認:花井,黒木 \par
    ~~~集金:野本,楠本 \par
  \end{tpbcol}&
  \begin{tpccol}
    K101集合
  \end{tpccol}&
  \begin{tpdcol}
    K101に来次第出席確認を行い,名札を受け取る \par
    参加費もその時払う \par
    荷物札に自分の名前を書き,荷物につける \par
    教室に新歓の物品でないものがないか確認する
  \end{tpdcol}\\

  \begin{tpacol}
    07:20
  \end{tpacol}&
  \begin{tpbcol}
    全員 \par
    ~~~説明:芦田
  \end{tpbcol}&
  \begin{tpccol}
    最終打ち合わせ
  \end{tpccol}&
  \begin{tpdcol}
    主に変更点の説明を行う \par
    不明な点があったら,この時間に解決させる  \par
    終わり次第,イベントの練習等を行う
  \end{tpdcol}\\


  \begin{tpacol}
    10:15
  \end{tpacol}&
  \begin{tpbcol}
    全員
  \end{tpbcol}&
  \begin{tpccol}
    受付準備開始
  \end{tpccol}&
  \begin{tpdcol}
    先遣隊は荷物搬入に向かう \par
=======
    09:00
    % 時間は4桁(〇〇:〇〇)で記入する
  \end{tpacol}&
  \begin{tpbcol}
    役割1
    % 役割or担当者を記入する
  \end{tpbcol}&
  \begin{tpccol}
    内容
  \end{tpccol}&
  \begin{tpdcol}
    備考(ない場合は空白でOK)
  \end{tpdcol}\\

  % 同じ時間でも担当が違う場合は別の行に記入する(その際の時間の記入の必要はなし)
  \begin{tpacol}
    
  \end{tpacol}&
  \begin{tpbcol}
    役割2
  \end{tpbcol}&
  \begin{tpccol}
    内容
  \end{tpccol}&
  \begin{tpdcol}
    備考 \par
    (表内改行可能です)
  \end{tpdcol}\\

  \begin{tpacol}
    
  \end{tpacol}&
  \begin{tpbcol}
    
  \end{tpbcol}&
  \begin{tpccol}
    
  \end{tpccol}&
  \begin{tpdcol}
    
>>>>>>> 0dc73cf63a5cf1767d6c1c4abd7880ac2b8de408
  \end{tpdcol}\\


\end{longtable}
\normalsize

\subsection{準備物品}
\begin{table}[h]
  \centering
  \caption{物品}
  \begin{tabular}{lrl}
    \hline
    \multicolumn{1}{c}{物品} & \multicolumn{1}{c}{数量} & \multicolumn{1}{c}{使用用途・備考} \\
    \hline
<<<<<<< HEAD
    荷物札 & 60 &  \\
    企画書 & 48 & 事前に印刷しておく \\
    \hline  
  \end{tabular}
\end{table}
=======
    物品A & 2 & 使用用途・備考(特に記載する必要なければ空白でOK) \\
    物品B & 11 &  \\
    \hline  
  \end{tabular}
\end{table}

\subsection{備考}
% 特に書くことなければ省略してOK
>>>>>>> 0dc73cf63a5cf1767d6c1c4abd7880ac2b8de408
