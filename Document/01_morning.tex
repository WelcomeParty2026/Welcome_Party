\section{朝の動き}

\subsection{場所・人員配置}
\begin{table}[h]
  \centering
  \begin{minipage}[t]{0.48\columnwidth}
    \centering
    \caption{場所・該当グループ}
    \begin{tabular}{cl}
      \hline
      場所 & K101 \\
      \hline
      グループ & 全員 \\
    % グループは2/10に分けたやつです.企画書班のNotionにも追加しておくので参考にしておいてください.
    \begin{tabular}{cl}
      \hline
      場所 & 体育館 \\
      \hline
      グループ & 先遣隊 \\
        & 後遣隊 \\
      \hline
    \end{tabular}
  \end{minipage}
  \begin{minipage}[t]{0.48\columnwidth}
    \centering
    \caption{役割分担}
    \begin{tabular}{ll}
      \hline
      \multicolumn{1}{c}{役割} & \multicolumn{1}{c}{担当者} \\
      \hline
      説明 & 人物A \\
      出席確認 & 人物B,人物C \\
      集金 & 人物D,人物E \\
      \hline
    \end{tabular}
  \end{minipage}
\end{table}

\subsection{詳細タイムスケジュール}
\vspace{0.5em}
\fontsize{8pt}{12pt}\selectfont

\begin{longtable}[c]{|p{.04\textwidth}|p{.2\textwidth}|p{.25\textwidth}|p{.45\textwidth}|}
  \caption{スケジュール} \\
  \hline
  \multicolumn{1}{|c|}{時間} & \multicolumn{1}{c|}{担当} & \multicolumn{1}{c|}{内容} & \multicolumn{1}{c|}{備考} \\
  \hline
  \endfirsthead
  
  \hline
  \multicolumn{1}{|c|}{時間} & \multicolumn{1}{c|}{担当} & \multicolumn{1}{c|}{内容} & \multicolumn{1}{c|}{備考} \\
  \hline
  \endhead
  
  \hline
  \endfoot
  
  \hline
  \endlastfoot
  
  \begin{tpacol}
    06:00
  \end{tpacol}&
  \begin{tpbcol}
    全員
  \end{tpbcol}&
  \begin{tpccol}
    各自起床
  \end{tpccol}&
  \begin{tpdcol}
    起床したらslackの\#当日報告に起きた旨を連絡する\par
    連絡がない場合は電話する
  \end{tpdcol}\\

  \begin{tpacol}
    07:00
  \end{tpacol}&
  \begin{tpbcol}
    全員 \par
    ~~~出席確認:人物A,人物B \par
    ~~~集金:人物C,人物D \par
  \end{tpbcol}&
  \begin{tpccol}
    K101集合
  \end{tpccol}&
  \begin{tpdcol}
    K101に来次第出席確認を行い,名札を受け取る \par
    参加費もその時払う \par
    荷物札に自分の名前を書き,荷物につける \par
    教室に新歓の物品でないものがないか確認する
  \end{tpdcol}\\

  \begin{tpacol}
    07:20
  \end{tpacol}&
  \begin{tpbcol}
    全員 \par
    ~~~説明:人物A
  \end{tpbcol}&
  \begin{tpccol}
    最終打ち合わせ
  \end{tpccol}&
  \begin{tpdcol}
    主に変更点の説明を行う \par
    不明な点があったら,この時間に解決させる  \par
    終わり次第,イベントの練習等を行う
  \end{tpdcol}\\


  \begin{tpacol}
    10:15
  \end{tpacol}&
  \begin{tpbcol}
    全員
  \end{tpbcol}&
  \begin{tpccol}
    受付準備開始
  \end{tpccol}&
  \begin{tpdcol}
    先遣隊は荷物搬入に向かう \par
  \end{tpdcol}\\


\end{longtable}
\normalsize

\subsection{準備物品}
\begin{table}[h]
  \centering
  \caption{物品}
  \begin{tabular}{lrl}
    \hline
    \multicolumn{1}{c}{物品} & \multicolumn{1}{c}{数量} & \multicolumn{1}{c}{使用用途・備考} \\
    \hline
    荷物札 & 60 &  \\
    企画書 & 48 & 事前に印刷しておく \\
    \hline  
  \end{tabular}
\end{table}
