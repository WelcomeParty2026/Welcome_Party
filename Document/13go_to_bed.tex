\newpage

\section{就寝}

\subsection{場所・人員配置}
\begin{table}[h]
  \centering
  \begin{minipage}[t]{0.48\columnwidth}
    \centering
    \caption{場所・該当グループ}
    \begin{tabular}{cl}
      \hline
      場所 & 宿泊棟 \\
      \hline
    \end{tabular}
  \end{minipage}
  \begin{minipage}[t]{0.48\columnwidth}
    \centering
    \caption{役割分担}
    \begin{tabular}{ll}
      \hline
      \multicolumn{1}{c}{役割} & \multicolumn{1}{c}{担当者} \\
      \hline
      宿泊棟B,C & , \\
      宿泊棟D &  \\
      \hline
    \end{tabular}
  \end{minipage}
\end{table}

\subsection{詳細タイムスケジュール}
\vspace{0.5em}
\fontsize{8pt}{12pt}\selectfont

\begin{longtable}[c]{|p{.04\textwidth}|p{.2\textwidth}|p{.25\textwidth}|p{.45\textwidth}|}
  \caption{スケジュール} \\
  \hline
  \multicolumn{1}{|c|}{時間} & \multicolumn{1}{c|}{担当} & \multicolumn{1}{c|}{内容} & \multicolumn{1}{c|}{備考} \\
  \hline
  \endfirsthead
  
  \hline
  \multicolumn{1}{|c|}{時間} & \multicolumn{1}{c|}{担当} & \multicolumn{1}{c|}{内容} & \multicolumn{1}{c|}{備考} \\
  \hline
  \endhead
  
  \hline
  \endfoot
  
  \hline
  \endlastfoot
  
  \begin{tpacol}
    22:00
  \end{tpacol}&
  \begin{tpbcol}
    見回り担当
  \end{tpbcol}&
  \begin{tpccol}
    第1回見回り
  \end{tpccol}&
  \begin{tpdcol}
    消灯する
  \end{tpdcol}\\

  \begin{tpacol}
    22:30
  \end{tpacol}&
  \begin{tpbcol}
    
  \end{tpbcol}&
  \begin{tpccol}
    第2回見回り
  \end{tpccol}&
  \begin{tpdcol}
    うるさいようであれば注意する
  \end{tpdcol}\\

\end{longtable}
\normalsize

\subsection{注意事項}
\begin{itemize}
  \item 自動販売機の使用は原則禁止する
  \item 万が一自動販売機へ行きたい新入生がいたら,自動販売機(食堂前と宿泊棟休憩室内)へ誘導し,その学生が買い終わるまで見張る(そのまま脱走する可能性を防ぐため)
  \item 女子棟に男子が,男子棟に女子が行かないように見張る
  \item それと同時に男性の先生方が女子棟に行かないように注意する
  \item その他何が起こるかわからないため,見回り時に何かあったらSlackの\#当日報告に連絡する
\end{itemize}

\subsection{備考}
\begin{itemize}
  \item 状況に応じて見回りの回数を増やす
  \item 車だし担当スタッフは優先的に就寝できるように配慮する
  \item 当日は懐中電灯を自分が就寝する棟に置いておき,必要に応じて持ち出す
  \item 何かあったらSlackの\#当日報告で報告する
\end{itemize}
