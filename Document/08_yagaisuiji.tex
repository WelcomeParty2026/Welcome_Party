\newpage

\section{野外炊事}

\subsection{場所・人員配置}
\begin{table}[h]
  \centering
  \begin{minipage}[t]{0.48\columnwidth}
    \centering
    \caption{場所・該当グループ}
    \begin{tabular}{cl}
      \hline
      場所 & 野外炊事場 \\
      \hline
      グループ & 全員 \\
      \hline
    \end{tabular}
  \end{minipage}
  \begin{minipage}[t]{0.48\columnwidth}
    \centering
    \caption{役割分担}
    \begin{tabular}{ll}
      \hline
      \multicolumn{1}{c}{役割} & \multicolumn{1}{c}{担当者} \\
      \hline
      鍵担当 & 人物A $\to$ 人物B \\
      統括 & 人物A \\
      撤収 & 人物A,人物B,人物C,人物D \\
       & 人物E,人物F,人物G,人物H \\
      \hline
    \end{tabular}
  \end{minipage}
\end{table}

\subsection{全体配置} 
\begin{figure}[htbp]
    \begin{center}
      \includegraphics[width = 10cm]{document/3_Deta/08yagaisuijizyo.pdf}
    \end{center}
  \end{figure}  

\newpage

\subsection{詳細タイムスケジュール}
\vspace{0.5em}
\fontsize{8pt}{12pt}\selectfont

\begin{longtable}[c]{|p{.04\textwidth}|p{.2\textwidth}|p{.25\textwidth}|p{.45\textwidth}|}
  \caption{スケジュール} \\
  \hline
  \multicolumn{1}{|c|}{時間} & \multicolumn{1}{c|}{担当} & \multicolumn{1}{c|}{内容} & \multicolumn{1}{c|}{備考} \\
  \hline
  \endfirsthead
  
  \hline
  \multicolumn{1}{|c|}{時間} & \multicolumn{1}{c|}{担当} & \multicolumn{1}{c|}{内容} & \multicolumn{1}{c|}{備考} \\
  \hline
  \endhead
  
  \hline
  \endfoot
  
  \hline
  \endlastfoot

  \begin{tpacol}
    16:00
  \end{tpacol}&
  \begin{tpbcol}
    班付
  \end{tpbcol}&
  \begin{tpccol}
    食材取りに行く
  \end{tpccol}&
  \begin{tpdcol}
    食堂前まで移動し,自分の班の人数の食材を受け取り,野外炊事場に移動する
  \end{tpdcol}\\

  \begin{tpacol}
   
  \end{tpacol}&
  \begin{tpbcol}
    全員
  \end{tpbcol}&
  \begin{tpccol}
    野外炊事オリエンテーション
  \end{tpccol}&
  \begin{tpdcol}
    室戸の方の指示に従う
  \end{tpdcol}\\

  \begin{tpacol}
    16:30
  \end{tpacol}&
  \begin{tpbcol}
    各班スタッフ
  \end{tpbcol}&
  \begin{tpccol}
    班内で自己紹介を行う
  \end{tpccol}&
  \begin{tpdcol}
    名前・ニックネーム・得意料理・意気込み
  \end{tpdcol}\\

  \begin{tpacol}
    16:35
  \end{tpacol}&
  \begin{tpbcol}
    全員
  \end{tpbcol}&
  \begin{tpccol}
    野外炊事開始
  \end{tpccol}&
  \begin{tpdcol}
    
  \end{tpdcol}\\

  \begin{tpacol}
    18:00
  \end{tpacol}&
  \begin{tpbcol}
    各班スタッフ
  \end{tpbcol}&
  \begin{tpccol}
    片付け開始、薪と食器の整理
  \end{tpccol}&
  \begin{tpdcol}
    職員チェック後、食器を戻す \par
    班付がチェックが終わった旨を総括に伝え,宿泊棟に戻る \par
    宿泊棟まで案内する間に,入浴場所,多目的ホールの場所も説明する
  \end{tpdcol}\\

  \begin{tpacol}
    
  \end{tpacol}&
  \begin{tpbcol}
    撤収
  \end{tpbcol}&
  \begin{tpccol}
    撤収作業
  \end{tpccol}&
  \begin{tpdcol}
    ゴミ処理と荷物の回収を行う\par
    食材トレーはいくつか重ねて食堂まで持って行く
  \end{tpdcol}\\

  \begin{tpacol}
    19:00
  \end{tpacol}&
  \begin{tpbcol}
    全員
  \end{tpbcol}&
  \begin{tpccol}
    完全撤退
  \end{tpccol}&
  \begin{tpdcol}
    総括は全般チェックが終わった旨を\#当日報告に連絡する
  \end{tpdcol}\\

\end{longtable}
\normalsize

\subsection{レシピ}
\begin{itembox}[l]{カレーのレシピ}
    \begin{enumerate}
      \item 野菜を綺麗に洗う.
      \item 野菜は大きすぎると火が通らないので細かく薄く切る.
      \item 肉は半分に切る.
      \item 切った具材を鍋に入れ、ひたひたになるまで水を入れる(水を入れすぎない).
      \item 火にかける.
      \item 沸騰まで放置する(スプーン等で刺し、野菜の柔らかさを確認する).
      \item ルーを開封前に細かく砕く.
      \item 混ぜながらルーを入れる.
      \item ルーが溶けたら終了.
    \end{enumerate}
  \end{itembox}
  
  \begin{itembox}[l]{お米の炊き方}
    \begin{enumerate}
      \item 米を研ぐ.
      \item 米の表面から一差し指第2関節に行かないぐらいまで水を入れる.
      \item 火にかける.
      \item 最初は中火(底に火がふれるくらい)、途中で強火(はんごう全体が火に包まれるくらい)で加熱してやると美味しいご飯ができる.
      \item 吹きこぼれて、吹きこぼれが乾いたら火から遠ざける(火にかけはじめて20分ほど待って吹きこぼれなかったらふたを開けて確認する).
      \item ご飯がべしゃべしゃの状態で炊けてしまったときはかまどの端っこの方(あまり火の当たらないところ)で1分ごとはんごうを回しながら水分を飛ばしてやると良い.
    \end{enumerate}
    ※吹きこぼれている時には、絶対にふたは空けない.
  \end{itembox}
  
\subsection{備考}
\begin{itemize}
    \item 火起こしの作業が遅れているときは他の班から人員を派遣する。
    \item 野外炊飯が早く終わった場合は次の工程に進む。
    \item 野外炊事時に鍵係を人物Aから人物Bに交代
\end{itemize}
