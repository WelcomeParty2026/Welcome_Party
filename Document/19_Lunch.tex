\newpage

\section{昼食・バス}  % 割り振りの名前を記入

\subsection{場所・人員配置}
\begin{table}[h]
  \centering
  \begin{minipage}[t]{0.48\columnwidth}
    \centering
    \caption{場所・該当グループ}
    % グループは2/10に分けたやつです.企画書班のNotionにも追加しておくので参考にしておいてください.
    \begin{tabular}{cl}
      \hline
      場所 & 食堂 \\
      & 中央広場 \\
      \hline
    \end{tabular}
  \end{minipage}
  \begin{minipage}[t]{0.48\columnwidth}
    \centering
    \caption{役割分担}
    \begin{tabular}{ll}
      \hline
      \multicolumn{1}{c}{役割} & \multicolumn{1}{c}{担当者} \\
      \hline
      配膳 & 大峯,田島,打越,細川 \\
      \hline
    \end{tabular}
  \end{minipage}
\end{table}

\subsection{詳細タイムスケジュール}
\vspace{0.5em}
\fontsize{8pt}{12pt}\selectfont

\begin{longtable}[c]{|p{.04\textwidth}|p{.2\textwidth}|p{.25\textwidth}|p{.45\textwidth}|}
  \caption{スケジュール} \\
  \hline
  \multicolumn{1}{|c|}{時間} & \multicolumn{1}{c|}{担当} & \multicolumn{1}{c|}{内容} & \multicolumn{1}{c|}{備考} \\
  \hline
  \endfirsthead
  
  \hline
  \multicolumn{1}{|c|}{時間} & \multicolumn{1}{c|}{担当} & \multicolumn{1}{c|}{内容} & \multicolumn{1}{c|}{備考} \\
  \hline
  \endhead
  
  \hline
  \endfoot
  
  \hline
  \endlastfoot
  
  \begin{tpacol}
    12:00
  \end{tpacol}&
  \begin{tpbcol}
    配膳係
  \end{tpbcol}&
  \begin{tpccol}
    配膳を行う
  \end{tpccol}&
  \begin{tpdcol}
    12:45までに全員食べ終わる
  \end{tpdcol}\\

  \begin{tpacol}
    12:45
  \end{tpacol}&
  \begin{tpbcol}
    全員
  \end{tpbcol}&
  \begin{tpccol}
    昼食終わり
  \end{tpccol}&
  \begin{tpdcol}
    忘れ物を確認する
  \end{tpdcol}\\

  \begin{tpacol}
    13:30
  \end{tpacol}&
  \begin{tpbcol}
    全員
  \end{tpbcol}&
  \begin{tpccol}
    中央広場集合
  \end{tpccol}&
  \begin{tpdcol}
    荷物を持った状態 \par
    人数を数え,揃ったところからバスに乗り込む \par
    アンケートのQRコードを渡す
  \end{tpdcol}\\
    
  \begin{tpacol}
    13:45
  \end{tpacol}&
  \begin{tpbcol}
    全員
  \end{tpbcol}&
  \begin{tpccol}
    バス出発
  \end{tpccol}&
  \begin{tpdcol}
    行きと同じ場所で休憩をとる
  \end{tpdcol}\\

\end{longtable}
\normalsize

\subsection{準備物品}
\begin{table}[h]
  \centering
  \caption{物品}
  \begin{tabular}{lrl}
    \hline
    \multicolumn{1}{c}{物品} & \multicolumn{1}{c}{数量} \\
    \hline
    乗車確認リスト & 4  \\
    司会者セット & 4 \\
    名札回収用袋 & 4 \\
    QRコード & 126 \\
    \hline  
  \end{tabular}
\end{table}