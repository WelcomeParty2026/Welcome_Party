\newpage

\section{本隊・後遣隊の動き} 

\subsection{場所・人員配置}
\begin{table}[h]
  \centering
  \begin{minipage}[t]{0.48\columnwidth}
    \centering
    \caption{場所・該当グループ}
    \begin{tabular}{cl}
      \hline
      場所 & K101 \\
      \hline
      グループ & 班付 \\
       & 裏方 \\
       & イベント運営 \\
       & 後遣隊 \\
      \hline
    \end{tabular}
  \end{minipage}
  \begin{minipage}[t]{0.48\columnwidth}
    \centering
    \caption{役割分担}
    \begin{tabular}{ll}
      \hline
      \multicolumn{1}{c}{役割} & \multicolumn{1}{c}{担当者} \\
      \hline
      受付 &  名前を入力 \\
      受付補助 &  名前を入力 \\
      先生方の受付 &  名前を入力 \\
      統括 &  名前を入力 \\
      誘導(K101) &  名前を入力 \\    
      バス対応 &  名前を入力 \\  
      誘導(バス) &  名前を入力 \\
      \hline
    \end{tabular}
  \end{minipage}
\end{table}

\subsection{詳細タイムスケジュール}
\vspace{0.5em}
\fontsize{8pt}{12pt}\selectfont

\begin{longtable}[c]{|p{.04\textwidth}|p{.2\textwidth}|p{.25\textwidth}|p{.45\textwidth}|}
  \caption{スケジュール} \\
  \hline
  \multicolumn{1}{|c|}{時間} & \multicolumn{1}{c|}{担当} & \multicolumn{1}{c|}{内容} & \multicolumn{1}{c|}{備考} \\
  \hline
  \endfirsthead
  
  \hline
  \multicolumn{1}{|c|}{時間} & \multicolumn{1}{c|}{担当} & \multicolumn{1}{c|}{内容} & \multicolumn{1}{c|}{備考} \\
  \hline
  \endhead
  
  \hline
  \endfoot
  
  \hline
  \endlastfoot
  
  \begin{tpacol}
    10:15
  \end{tpacol}&
  \begin{tpbcol}
    
  \end{tpbcol}&
  \begin{tpccol}
    受付準備開始
  \end{tpccol}&
  \begin{tpdcol}
    
  \end{tpdcol}\\

  \begin{tpacol}
    
  \end{tpacol}&
  \begin{tpbcol}
    統括
  \end{tpbcol}&
  \begin{tpccol}
    ホワイトボードに諸注意を書く
  \end{tpccol}&
  \begin{tpdcol}
    備考欄参考のこと
  \end{tpdcol}\\

  \begin{tpacol}
    
  \end{tpacol}&
  \begin{tpbcol}
    受付,受付補助 \par
    先生方の受付
  \end{tpbcol}&
  \begin{tpccol}
    受付準備
  \end{tpccol}&
  \begin{tpdcol}
    名簿,ボールペン,名札,集金ボックス,お釣り用のお金を \par
    机に並べる
  \end{tpdcol}\\

  \begin{tpacol}
    
  \end{tpacol}&
  \begin{tpbcol}
    手の空いているスタッフ
  \end{tpbcol}&
  \begin{tpccol}
    受付準備の手伝いをする
  \end{tpccol}&
  \begin{tpdcol}
    大きなゴミや汚れを見つけた場合は掃除する
  \end{tpdcol}\\

  \begin{tpacol}
    11:00
  \end{tpacol}&
  \begin{tpbcol}
    
  \end{tpbcol}&
  \begin{tpccol}
    受付開始
  \end{tpccol}&
  \begin{tpdcol}
    
  \end{tpdcol}\\

  \begin{tpacol}
    
  \end{tpacol}&
  \begin{tpbcol}
    受付,受付補助 \par
    先生方の受付
  \end{tpbcol}&
  \begin{tpccol}
    新入生,先生方の受付を行う
  \end{tpccol}&
  \begin{tpdcol}
    備考欄参考のこと
  \end{tpdcol}\\

  \begin{tpacol}
    
  \end{tpacol}&
  \begin{tpbcol}
    誘導(K101)
  \end{tpbcol}&
  \begin{tpccol}
    K101の近くで新入生を案内する
  \end{tpccol}&
  \begin{tpdcol}
    必ず南側の入り口に誘導し,学籍番号でブースが分かれていることを伝える
  \end{tpdcol}\\

  \begin{tpacol}
    
  \end{tpacol}&
  \begin{tpbcol}
    統括
  \end{tpbcol}&
  \begin{tpccol}
    全体の見回り
  \end{tpccol}&
  \begin{tpdcol}
    何かトラブルがあった場合は統括の判断に従う
  \end{tpdcol}\\

  \begin{tpacol}
    
  \end{tpacol}&
  \begin{tpbcol}
    手の空いているスタッフ
  \end{tpbcol}&
  \begin{tpccol}
    新入生と積極的に関わる
  \end{tpccol}&
  \begin{tpdcol}
    スタッフ同士で喋らないこと
  \end{tpdcol}\\

  \begin{tpacol}
    11:40
  \end{tpacol}&
  \begin{tpbcol}
    バス対応
  \end{tpbcol}&
  \begin{tpccol}
    到着したバスに挨拶する
  \end{tpccol}&
  \begin{tpdcol}
    少し前に自分の荷物とバス司会用物品を持って東ロータリーに移動し,バスが来るのを待っておく. \par
    バスが来たら\#当日報告に連絡する \par
    確認事項を確認しておく(備考欄参考のこと)
  \end{tpdcol}\\

  \begin{tpacol}
    11:45
  \end{tpacol}&
  \begin{tpbcol}
    後遣隊
  \end{tpbcol}&
  \begin{tpccol}
    遅れてきた新入生の受付対応
  \end{tpccol}&
  \begin{tpdcol}
    バスに間に合うようであればバスに,間に合わないようであれば車に乗せる
  \end{tpdcol}\\

  \begin{tpacol}
    
  \end{tpacol}&
  \begin{tpbcol}
    受付,受付補助\par
    先生方の受付
  \end{tpbcol}&
  \begin{tpccol}
    各ブースごとに受付を済ませた \par
    人数をまとめる
  \end{tpccol}&
  \begin{tpdcol}
    まだ来ていない新入生には電話で連絡する \par
    集計の結果をバス対応係に伝える
  \end{tpdcol}\\

  \begin{tpacol}
    
  \end{tpacol}&
  \begin{tpbcol}
    誘導(バス)
  \end{tpbcol}&
  \begin{tpccol}
    新入生・先生方の誘導
  \end{tpccol}&
  \begin{tpdcol}
    バス到着の連絡を受け次第,自分の荷物を持って誘導する \par
    混雑を防ぐため,プラカードを持って1号車から順に誘導する \par
    体調不良やバス酔いがある方は前に座るように伝える \par
    先生方の荷物をお預かりしていた場合は確実に運ぶ
  \end{tpdcol}\\

  \begin{tpacol}
    
  \end{tpacol}&
  \begin{tpbcol}
    手の空いているスタッフ
  \end{tpbcol}&
  \begin{tpccol}
    掃除,忘れ物チェック
  \end{tpccol}&
  \begin{tpdcol}
    目に見えるゴミは確実に捨てる \par
    忘れ物は\#当日報告に連絡し,バスの司会者が新入生・先生方に確認をとる \par
    終わり次第バスに乗り込む
  \end{tpdcol}\\

  \begin{tpacol}
    12:00
  \end{tpacol}&
  \begin{tpbcol}
    バス対応
  \end{tpbcol}&
  \begin{tpccol}
    人数確認
  \end{tpccol}&
  \begin{tpdcol}
    バスの入り口付近で乗車確認リストを用いて乗車する人のチェックを行う \par
    奥から詰めて座るよう伝える \par
    全員乗車したら,人数を確認する
  \end{tpdcol}\\

  \begin{tpacol}
    
  \end{tpacol}&
  \begin{tpbcol}
    
  \end{tpbcol}&
  \begin{tpccol}
    受付人数と乗車人数が合うかを確認する
  \end{tpccol}&
  \begin{tpdcol}
    受付済みだがバスに乗車していない場合は大学内を探す \par
    バスの乗車リストにはチェックがあるが,受付済みでない場合はその人が本当にバスに乗車しているかを再確認する
  \end{tpdcol}\\

  \begin{tpacol}
    12:10
  \end{tpacol}&
  \begin{tpbcol}
    本隊全員
  \end{tpbcol}&
  \begin{tpccol}
    バス出発
  \end{tpccol}&
  \begin{tpdcol}
    この時間までに全員バスに乗車する \par
    出発した旨を\#当日報告に連絡する(名前を入力)
  \end{tpdcol}\\

  \begin{tpacol}
    12:30
  \end{tpacol}&
  \begin{tpbcol}
    後遣隊
  \end{tpbcol}&
  \begin{tpccol}
    受付完全終了
  \end{tpccol}&
  \begin{tpdcol}
    参加費の計算を行い,合うか確認する \par
    計算があった場合は受付物品と参加費を福本研究室に運ぶ \par
    K101の電気を消すのを忘れないこと
  \end{tpdcol}\\

  \begin{tpacol}
    13:00
  \end{tpacol}&
  \begin{tpbcol}
    
  \end{tpbcol}&
  \begin{tpccol}
    出発準備・出発
  \end{tpccol}&
  \begin{tpdcol}
    K101の撤収が終わり次第,忘れ物を車に運んで出発する \par
    バスに乗れなかった新入生も乗せる \par
    出発した旨を\#当日報告に連絡する \par
    残りの買い出しを行いながら室戸に向かう
  \end{tpdcol}\\

\end{longtable}
\normalsize

\subsection{準備物品}
\begin{table}[h]
  \centering
  \caption{物品}
  \begin{tabular}{lrl}
    \hline
    \multicolumn{1}{c}{物品} & \multicolumn{1}{c}{数量} & \multicolumn{1}{c}{使用用途・備考} \\
    \hline
    名札 & 154 & 受付時に渡す \\
    受付名簿 & 5 & 受付用 \\
    ボールペン & 5 & 受付用 \\
    集金ボックス & 5 & 受付用 \\
    お釣り用のお金 & 5セット & 受付用 \\
    バス名簿 & 4 & バス乗車用 \\
    荷札 & 40 & 大きい荷物の人用 \\
    プラカード & 1セット & バス誘導用 \\
    乗車確認リスト & 1セット & バス人数確認用 \\
    バス司会物品 & 4セット & 酔い止め,エチケット袋,紙コップ,水 \\  
    \hline  
  \end{tabular}
\end{table}

\newpage

\subsection{備考}
ホワイトボードに書く諸注意
\begin{itemize}
  \item バスは11:45に乗り込むのでそれまでにトイレを済ませておく
  \item バスが来るまで荷物は各自で持っておき,そのままバスの各自座席に持って行く
  \item バスの号車,野外炊事,就寝部屋,イベントのグループは名札に書いてある
  \item トイレに行く場合はスタッフに声をかけてから行くこと
  \item 必要のある場合以外は部屋の外に出ない
  \item 教室から出る時に忘れ物がないか確認する
  \item 受付を済ませた新入生は名札を身につけ,しおりの6ページ目にある自己紹介カードを記入する
  \item 周りの新入生やスタッフと自由に話してOK
  \item 前から詰めて座って欲しい
\end{itemize}

新入生の受付対応
\begin{itemize}
  \item 名前を聞いて名簿のチェック欄にチェックをして参加費(500円)を徴収する
  \item 新入生に名札を渡す
  \item 新入生の荷物のうち,トランクに入れる必要がある大きい荷物は,新入生にタグを渡して名前を書き,つけてもらう
  \item その後,各バスの席を案内し,ホワイトボードの諸注意を読んでおくように伝える
  \item アレルギーがある人が来たらその旨を田中に伝える
  \item 大きい荷物の判断は受付担当が行う
  \item お金が払えない人はチェックしておき,後日必ず回収する
\end{itemize}

先生方の受付対応
\begin{itemize}
  \item 名簿のチェック欄にチェックをする(参加費は徴収しない)
  \item カンパをもらっていない先生にはカンパを頂戴する
  \item 先生に名札を渡す
  \item 先生に荷物は自分で持たれるかスタッフが運ぶかお伺いし,スタッフが運ぶ場合は名前を書いたタグをつけて荷物を受け取る
  \item 受付が終了した先生はK101で待機をお願いする
\end{itemize}

バスの確認事項
\begin{itemize}
  \item バス内でのマイクの有無,使用について
  \item 休憩場所,到着時間の確認
  \begin{itemize}
    \item 1号車,2号車:安芸駅
    \item 3号車,4号車:田野駅
  \end{itemize}
\end{itemize}

その他
\begin{itemize}
  \item 北側の扉は閉めておくこと
  \item トイレに行く報告を受けた場合はメモをとり,帰ってきた場合も一声かけて欲しい旨を伝える
\end{itemize}