\newpage

\section{退所式・記念撮影}  % 割り振りの名前を記入

\subsection{場所・人員配置}
\begin{table}[h]
  \centering
  \begin{minipage}[t]{0.48\columnwidth}
    \centering
    \caption{場所・該当グループ}
    % グループは2/10に分けたやつです.企画書班のNotionにも追加しておくので参考にしておいてください.
    \begin{tabular}{cc}
      \hline
      場所 & 体育館 \\
      \hline
      グループ & 全員 \\
      \hline
    \end{tabular}
  \end{minipage}
  \begin{minipage}[t]{0.48\columnwidth}
    \centering
    \caption{役割分担}
    \begin{tabular}{ll}
      \hline
      \multicolumn{1}{c}{役割} & \multicolumn{1}{c}{担当者} \\
      \hline
      司会 & 人物1 \\
      挨拶 & 芦田 \\
      \hline
    \end{tabular}
  \end{minipage}
\end{table}

\subsection{詳細タイムスケジュール}
\vspace{0.5em}
\fontsize{8pt}{12pt}\selectfont

\begin{longtable}[c]{|p{.04\textwidth}|p{.2\textwidth}|p{.25\textwidth}|p{.45\textwidth}|}
  \caption{スケジュール} \\
  \hline
  \multicolumn{1}{|c|}{時間} & \multicolumn{1}{c|}{担当} & \multicolumn{1}{c|}{内容} & \multicolumn{1}{c|}{備考} \\
  \hline
  \endfirsthead
  
  \hline
  \multicolumn{1}{|c|}{時間} & \multicolumn{1}{c|}{担当} & \multicolumn{1}{c|}{内容} & \multicolumn{1}{c|}{備考} \\
  \hline
  \endhead
  
  \hline
  \endfoot
  
  \hline
  \endlastfoot
  
  \begin{tpacol}
    11:00
  \end{tpacol}&
  \begin{tpbcol}
    司会
  \end{tpbcol}&
  \begin{tpccol}
    座談会終了
  \end{tpccol}&
  \begin{tpdcol}
    トイレ休憩のアナウンスを行い,11:15に退所式を始めことを伝える
  \end{tpdcol}\\

  \begin{tpacol}
    
  \end{tpacol}&
  \begin{tpbcol}
    芦田
  \end{tpbcol}&
  \begin{tpccol}
    室戸職員の方を呼びに行く
  \end{tpccol}&
  \begin{tpdcol}
    11:15から退所式を行うことを伝え,記念撮影をお願いする
  \end{tpdcol}\\

  \begin{tpacol}
    11:10
  \end{tpacol}&
  \begin{tpbcol}
    班付
  \end{tpbcol}&
  \begin{tpccol}
    プラカードを持ち一列に整列させ座らせる \par
    全員が揃ったら司会に報告する
  \end{tpccol}&
  \begin{tpdcol}
    
  \end{tpdcol}\\

  \begin{tpacol}
    11:15
  \end{tpacol}&
  \begin{tpbcol}
    司会
  \end{tpbcol}&
  \begin{tpccol}
    退所式
  \end{tpccol}&
  \begin{tpdcol}
    
  \end{tpdcol}\\

  \begin{tpacol}
    11:20
  \end{tpacol}&
  \begin{tpbcol}
    挨拶
  \end{tpbcol}&
  \begin{tpccol}
    代表挨拶
  \end{tpccol}&
  \begin{tpdcol}
    
  \end{tpdcol} \\

  \begin{tpacol}
    11:25
  \end{tpacol}&
  \begin{tpbcol}
    施設職員
  \end{tpbcol}&
  \begin{tpccol}
    挨拶
  \end{tpccol}&
  \begin{tpdcol}
    
  \end{tpdcol}\\

  \begin{tpacol}
    11:30
  \end{tpacol}&
  \begin{tpbcol}
    全員
  \end{tpbcol}&
  \begin{tpccol} 
    記念撮影
  \end{tpccol}&
  \begin{tpdcol}
    室戸職員の方にとっていただく
  \end{tpdcol}\\

  \begin{tpacol}
    11:35
  \end{tpacol}&
  \begin{tpbcol}
    司会
  \end{tpbcol}&
  \begin{tpccol} 
    12:00から昼食で,その後は13:30に荷物を持って中央広場に集合することを伝える
  \end{tpccol}&
  \begin{tpdcol}
    
  \end{tpdcol}\\

\end{longtable}
\normalsize

\subsection{準備物品}
\begin{table}[h]
  \centering
  \caption{物品}
  \begin{tabular}{lr}
    \hline
    \multicolumn{1}{c}{物品} & \multicolumn{1}{c}{数量}\\
    \hline
    プラカード & 20\\
    \hline  
  \end{tabular}
\end{table}
