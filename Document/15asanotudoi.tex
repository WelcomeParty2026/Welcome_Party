\newpage

\section{朝の集い}

\subsection{日時・場所}
\begin{description}
  \item[日時 :] 2025年4月7日(月) 7:00 $\sim$ 7:45
  \item[場所 :] 正面広場(雨天中止)
\end{description}

\subsection{タイムスケジュール(晴天時)}
\begin{description}
  \item[07:10] \textbf{移動開始}
  \begin{itemize}
    \item 整列係は新入生の誘導開始前に正面広場に移動し,整列の準備をする
    \item 誘導係は宿泊棟から誘導を開始する
    \item ?は誘導開始時にSlackの\#当日報告に報告する
    \item その他のスタッフは正面広場に到着後,誘導係,整列係の指示に従って新入生を並ばせる
  \end{itemize}

  \item[07:30] \textbf{国旗・施設旗掲揚 ラジオ体操 スタッフ代表の挨拶}
  \begin{itemize}
    \item イベント準備スタッフは朝のつどいが始まる頃に食堂に移動し,朝食をとる
    \item ?職員の指示に従い,国旗・施設旗掲揚,ラジオ体操(?),スタッフ代表の挨拶を行う
    \item ?はラジオ体操が終わり次第,朝食をとりにいく
    \item 朝食時の食堂内誘導スタッフは,朝のつどいの終わり頃に食堂へ移動しやすい場所で待機する

    ※野外炊事統括者は前日の野外炊事に国旗・所旗掲揚する新入生2名を見つけて交渉しておく\\
    ※国旗・所旗掲揚する新入生2名が見つからなければ,?が担当する
  \end{itemize}

  \item[07:40] \textbf{食堂へ移動開始}
  \begin{itemize}
    \item ?は挨拶終了後,アレルギーを持っている新入生は正面広場に残り,それ以外の新入生は食堂へ移動するように指示し,Slackの\#当日報告に食堂へ移動開始の旨を報告する
    \item ?はアレルギーを持っている新入生に対して,朝食・昼食をとる際の注意事項を伝え,食堂に移動する
    \item ?は先に食堂に向かい,配膳の準備を行う
    \item 最初に教職員から食堂へ行っていただくため,先生誘導係は食堂までの誘導を行う(新入生は待機)
    \item 食堂内誘導スタッフも急いで食堂に向かい,食堂内の各々のポジションに配置し,新入生が入ってきたら,奥から詰めて座ることを伝える
    \item 各宿泊棟から正面広間への誘導係と正面広間から食堂までの誘導係は食堂へ誘導を行う
    \item ?を先頭に食堂への誘導を開始し,?はトイレに行きたい新入生がいたら連れて行く (トイレは浴場横のものを使用する)
    \item 松尾は新入生の列の中間あたりで誘導を行い,食堂に着き次第,食堂内での誘導を行う
    \item 各スタッフも誘導しながら,奥からつめて座るように指示する
    \item ?は新入生の最後尾について全員の移動を確認する
  \end{itemize}

  \item[07:45] \textbf{食堂への移動完了}
  \begin{itemize}
    \item ?は全員が食堂に入り終えたらSlackの\#当日報告に報告する (?が新入生のトイレの行き帰りを誘導しているか確認)
  \end{itemize}
\end{description}

\subsection{タイムスケジュール(雨天時)}
\begin{description}
  \item[07:00] \textbf{朝のつどい中止アナウンス}
  \begin{itemize}
    \item 青少年の家から放送で朝の集い中止のアナウンスがある
    \item 中止の場合は引き続き清掃を行う
    \item ?は各宿泊棟をまわり,アレルギーを持っている新入生に対して,朝食・昼食をとる際の注意事項を伝える
  \end{itemize}

  \item[07:30] \textbf{食堂へ移動開始}
  \begin{itemize}
    \item イベント準備スタッフは早めに食堂に移動し,朝食をとる
    \item 誘導係は宿泊棟から食堂まで新入生と先生の誘導する
    \item ?は誘導開始時にSlackの\#当日報告に報告する
    \item ?,?は先に食堂に向かい,配膳の準備を行う
    \item 朝食時の食堂内誘導スタッフは食堂内の各々のポジションに配置し,新入生が入ってきたら,奥から詰めて座ることを伝える
    \item その他のスタッフは誘導の補助をしつつ食堂に移動する
  \end{itemize}

  \item[07:45] \textbf{食堂へ移動完了}
  \begin{itemize}
    \item 食堂内スタッフは入り口で誘導を行い,全員が食堂に入り終えたら塚脇がSlackの\#当日報告に報告する
  \end{itemize}
\end{description}

\subsection{人員配置}% イベント班以外
\begin{itemize}
  \item[◎] 晴天時
  \begin{itemize}
    \item 宿泊棟Aから正面広間への誘導係:?
    \item 宿泊棟Bから正面広間への誘導係:?
    \item 宿泊棟Cから正面広間への誘導係:?
    \item 宿泊棟Dから正面広間への誘導係:?
    \item  整列係:?
    \item ラジオ体操のお兄さん :?
    \item 旗揚げ係:スカウトした新入生2名
    \item スタッフ代表の挨拶:?
    \item 正面広間から食堂までの誘導係:?
    \item 食堂内での誘導係:?
    \item イベント準備スタッフ:?
  \end{itemize}
  \item[◎] 雨天時
  \begin{itemize}
    \item 宿泊棟Aから食堂への誘導係:?
    \item 宿泊棟Bから食堂への誘導係:?
    \item 宿泊棟Cから食堂への誘導係:?
    \item 宿泊棟Dから食堂への誘導係:?
    \item 食堂内での誘導係:?
    \item イベント準備スタッフ:?
  \end{itemize}
\end{itemize}

% 画像の配置が気持ち悪いので,画像を差し替えるときに配置を調整すること!!!!
\subsection{全体配置}
% \begin{figure}[htbp]
%   \centering
%   \includegraphics[scale=0.5]{document/asanotudoihaiti.eps}
%   \caption{正面広場}
%   \label{fig:hiroba}
% \end{figure}

% \subsection{朝のつどいから食堂までの誘導位置}
% \begin{figure}[htbp]
%   \centering
%   % \includegraphics[scale=0.5]{document/asanotudoihaiti.eps}
%   \caption{人員が決まり次第,マップと誘導者の配置を表した図を張る}
%   \label{fig:yudo}
% \end{figure}

\subsection{備考}
\begin{itemize}
  \item この時点で自分の部屋の片付けが終わっている場合,荷物を持って朝の集いに参加する.
\end{itemize}
